%_ PACKAGES __________________________________________________________________________ %

    %__ INPUT/OUTPUT LANGUAGE _________________________________ %
    \usepackage[brazilian]{babel}
	\usepackage[utf8]{inputenc}
	\usepackage{lmodern}
	\usefonttheme{serif}

    %__ MATH __________________________________________________ %
    \usepackage{amssymb}
    \usepackage{amsmath}
    \usepackage{amsthm}
    \usepackage{bbm}
    \usepackage{amsfonts}
    \DeclareMathOperator*{\argmax}{argmax}
    
    \usepackage{cancel}

    %__ GRAPHS & TABLES________________________________________ %
    \usepackage{graphicx}
    \usepackage{booktabs}
    \usepackage{multirow}
    \usepackage{array}
    \usepackage{lscape}
    \usepackage{caption}
    \usepackage{tabularx}
    \usepackage{tabularx}
        \newcolumntype{Z}{>{\centering\arraybackslash}X}
        \newcolumntype{L}{>{\raggedright\arraybackslash}X}	
    
        \renewcommand{\arraystretch}{1.2}
%        \usepackage{subcaption}			% SUBFIGURE IS DEPRECATED; SHOULD USE SUBCAPTION INSTEAD OF .
    \usepackage[flushleft,online,para]{threeparttable}
    \usepackage{multirow}
    \usepackage{dcolumn}
        \newcolumntype{d}[1]{D{.}{\cdot}{#1}}
    \usepackage{rotating}                                      % for **sideways**tables

    %\usepackage[nolists]{endfloat}     %   PUTS FIGURES AT THE END OF DOCUMENT.
                                        %   DOESN'T WORK WITH \usepackage{float}


%	\usepackage{ulem}

  \usepackage{listings}
  \usepackage{xcolor}
      \definecolor{BackgroundGray}{cmyk}{0,0,0,.1}
      \definecolor{CommentGreen}{cmyk}{.26,0,.50,.68}
      \definecolor{CPIgraydark}{RGB}{104,102,100}
      \definecolor{black}{rgb}{0.1,0.1,0.1}
      \definecolor{pcolor}{rgb}{0.25,0.29,0.3}
        
\setbeamercolor{frametitle}{fg=black,bg=white}
\setbeamercolor{title}{fg=black,bg=white}

  \lstset{ %
    backgroundcolor=\color{BackgroundGray}, % choose the background color; you must add \usepackage{color} or \usepackage{xcolor}
    basicstyle=\ttfamily\tiny,                 % the size of the fonts that are used for the code
    breakatwhitespace=false,              % sets if automatic breaks should only happen at whitespace
    breaklines=true,                      % sets automatic line breaking
    captionpos=b,                         % sets the caption-position to bottom
    commentstyle=\color{CommentGreen},    % comment style
    deletekeywords={...},                 % if you want to delete keywords from the given language
    escapechar={\@},                      % if you want to add LaTeX within your code
    extendedchars=true,                   % lets you use non-ASCII characters; for 8-bits encodings only, does not work with UTF-8
    frame=none,                           % adds a frame around the code
    keepspaces=true,                      % keeps spaces in text, useful for keeping indentation of code (possibly needs columns=flexible)
    keywordstyle=\color{blue},           % keyword style
    language=R,                           % the language of the code
    morekeywords={*,...},                 % if you want to add more keywords to the set
    morestring=[b]",
    morestring=[d]',
    numbers=none,                         % where to put the line-numbers; possible values are (none, left, right)
    numbersep=5pt,                        % how far the line-numbers are from the code
    numberstyle=\tiny\color{gray},        % the style that is used for the line-numbers
    rulecolor=\color{black},              % if not set, the frame-color may be changed on line-breaks within not-black text (e.g. comments (green here))
    showspaces=false,                     % show spaces everywhere adding particular underscores; it overrides 'showstringspaces'
    showstringspaces=false,               % underline spaces within strings only
    showtabs=false,                       % show tabs within strings adding particular underscores
    stepnumber=2,                         % the step between two line-numbers. If it's 1, each line will be numbered
    stringstyle=\color{red},              % string literal style
    tabsize=2,                            % sets default tabsize to 2 spaces
    title=\lstname,                       % show the filename of files included with \lstinputlisting; also try caption instead of title
    columns=flexible
  }
\lstset{literate=
  {á}{{\'a}}1 {é}{{\'e}}1 {í}{{\'i}}1 {ó}{{\'o}}1 {ú}{{\'u}}1
  {Á}{{\'A}}1 {É}{{\'E}}1 {Í}{{\'I}}1 {Ó}{{\'O}}1 {Ú}{{\'U}}1
  {à}{{\`a}}1 {è}{{\`e}}1 {ì}{{\`i}}1 {ò}{{\`o}}1 {ù}{{\`u}}1
  {À}{{\`A}}1 {È}{{\'E}}1 {Ì}{{\`I}}1 {Ò}{{\`O}}1 {Ù}{{\`U}}1
  {ä}{{\"a}}1 {ë}{{\"e}}1 {ï}{{\"i}}1 {ö}{{\"o}}1 {ü}{{\"u}}1
  {Ä}{{\"A}}1 {Ë}{{\"E}}1 {Ï}{{\"I}}1 {Ö}{{\"O}}1 {Ü}{{\"U}}1
  {â}{{\^a}}1 {ê}{{\^e}}1 {î}{{\^i}}1 {ô}{{\^o}}1 {û}{{\^u}}1
  {Â}{{\^A}}1 {Ê}{{\^E}}1 {Î}{{\^I}}1 {Ô}{{\^O}}1 {Û}{{\^U}}1
  {Ã}{{\~A}}1 {ã}{{\~a}}1
  {œ}{{\oe}}1 {Œ}{{\OE}}1 {æ}{{\ae}}1 {Æ}{{\AE}}1 {ß}{{\ss}}1
  {ç}{{\c c}}1 {Ç}{{\c C}}1 {ø}{{\o}}1 {å}{{\r a}}1 {Å}{{\r A}}1
  {€}{{\EUR}}1 {£}{{\pounds}}1
}  
  %\DefineShortVerb{\|}

\setbeamertemplate{itemize item}{\color{black}$\blacktriangleright$}
\setbeamertemplate{itemize subitem}{\color{black}$\blacksquare$}

\setbeamercolor{enumerate item}{fg=black,bg=white}
\setbeamercolor{enumerate subitem}{fg=black,bg=white}

    %__ BIBLIOGRAPHY __________________________________________ %
    \usepackage[round,longnamesfirst]{natbib}

    %__ APPENDIX _____________________________________________ %
%    \usepackage[titletoc,page]{appendix}

    %__ PDF, DISPLAY & PRODUCTIVITY ___________________________ %
    \usepackage{hyperref}
    %\usepackage[textsize=footnotesize,colorinlistoftodos,textwidth=4cm,obeyDraft]{todonotes}
    \usepackage{pgfpages}
    \usepackage{pdfpages}    
%    \usepackage{pdfsync}
    \usepackage{pifont} % wtf?
%    \usepackage{bbding} % wtf?
    \usepackage{verbatim}

    \usepackage{wasysym} % smiley/frownie


%%%%%%%%%%%%%%%%%%%%%%%%%%%%%%%%%%%%%%%%%%%%%%%%%%%%%%%%%%%%%%%%%%%%%%%%%%%%

\beamertemplatenavigationsymbolsempty

    \newcommand{\mc}{\multicolumn}
    \newcommand{\lbar}{\underline}
    \newcommand{\ubar}{\overline}

    \newtheorem{resposta}{Resposta}
    \newtheorem{pergunta}{Pergunta}    


%%%%%%%%%%%%%%%%%%%%%%%%%%%%%%%%%%%%%%%%%%%%%%%%%%%%%%%%%%%%%%%%%%%%%%%%%%%%

%\AtBeginSection[] {
%  \begin{frame}[plain]
%    \frametitle{Chapters}
%    \tableofcontents[currentsection]
%  \end{frame}
%  \addtocounter{totalframenumber}{-1}
%}

\addtocounter{framenumber}{-1}
% FOOTLINE - PAGE NUMBER RIGHT
\defbeamertemplate*{footline}{guildford foot theme}
{
  \leavevmode%
  \hbox{%
  \begin{beamercolorbox}[wd=.7\paperwidth,ht=1cm,dp=0ex,left]{}%
    {
    \insertsectionnavigationhorizontal{.5\paperwidth}{}{}
    }
 \end{beamercolorbox}
 \begin{beamercolorbox}[wd=0.31\paperwidth,ht=1cm,dp=0ex,right]{}%
	{\tiny
	\insertframenumber{} / \inserttotalframenumber\hspace*{5ex}
	}
 \end{beamercolorbox}}%
  \vskip5pt%
}

% Titles will appear in Small Cap Serif

\usefonttheme{structuresmallcapsserif}

% -----------------------------------------
% Center the Frame Title
% -----------------------------------------

\setbeamertemplate{frametitle} {
\begin{centering}
\vspace{0.1in} \insertframetitle
\par
\end{centering}}

% -----------------------------------------
% Number the slides
% -----------------------------------------

\setbeamertemplate{footline}[frame number]

% -----------------------------------------
% Get rid of the irritating navigation bar
% -----------------------------------------

\setbeamertemplate{navigation symbols}{}

%------------------------------------------

\newenvironment{wideitemize}{\itemize\addtolength{\itemsep}{8pt}}{\enditemize}

\newcommand{\ind}{\perp\!\!\!\!\perp}

%%%%%%%%%%%%%%%%%%%%%%%%%%%%%%%%%%%%%%%%%%%%%%%%%%%%%%%%%%%%%%%%%%%%%%%%%%%%
