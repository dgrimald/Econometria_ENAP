%_ PACKAGES __________________________________________________________________________ %
    
    %__ INPUT/OUTPUT LANGUAGE _________________________________ %
    \usepackage[brazilian]{babel}
    \usepackage[utf8]{inputenc}
    \usepackage[T1]{fontenc}
    %\usepackage{indentfirst}
    
    %__ MATH __________________________________________________ %
    \usepackage{amsfonts}
    \usepackage{amssymb}
    \usepackage{amsmath}
    \usepackage{amsthm}
    \usepackage{bbm}
    
    %__ GRAPHS & TABLES________________________________________ %
    \usepackage{graphicx}
    \usepackage{booktabs}
    \usepackage{multirow}
    \usepackage{array}    
    \usepackage{lscape}
    \usepackage{caption}
    \usepackage{subcaption}
    \usepackage[flushleft,online,para]{threeparttable}
    \usepackage{multirow}
    
    \usepackage{parskip}                       % WHAT IS THIS FOR?
    \usepackage{gastex}                        % WHAT IS THIS FOR?
    \usepackage[nolists]{endfloat}             % PUT FIGURES AT THE END OF DOCUMENT; DOESN'T WORK WITH \usepackage{float}
    
    \usepackage{floatrow}
        \floatsetup[table]{style=plaintop}     % LEAVE TABLE CAPTIONS AT THE TOP
        \floatsetup[figure]{style=plaintop}     % LEAVE TABLE CAPTIONS AT THE TOP
    
    \usepackage{dcolumn}
        \newcolumntype{d}[1]{D{.}{\cdot}{#1}}
    
    \usepackage{rotating}                      % for **sideways**tables
    
    %__ BIBLIOGRAPHY __________________________________________ %
    \usepackage[round,longnamesfirst]{natbib}
    %\usepackage[alf,abnt-and-type=e,abnt-etal-list=2]{abntcite}
    
    %__ PDF, DISPLAY & PRODUCTIVITY ___________________________ %
    \usepackage{xcolor}
        \definecolor{darkblue}{rgb}{0,0,0.5}
    \usepackage{hyperref}
        \hypersetup{
            colorlinks = true,
            linkcolor = darkblue,
            citecolor = darkblue,
            pdfborder = 0 0 0,
            pdfdisplaydoctitle = true,
            pdfhighlight = /N,
            pdfpagelayout = OneColumn,
            pdfpagemode = UseNone,
            pdfstartview = {FitH},
            pdfauthor = {{AB \& VP}},
            pdftitle = {{MPAM 2022}}
        }
        
    \usepackage{geometry}
        \geometry{verbose,tmargin=2.0cm,bmargin=2.0cm,lmargin=2.0cm,rmargin=2.0cm}
    \usepackage{setspace}
        \onehalfspacing
    
    \usepackage[bottom, multiple]{footmisc}    % keep footnotes at the bottom of the page, and allow for multiple footnotes at one place.
    
    \usepackage{verbatim}
    \usepackage[normalem]{ulem}     % strikethrough fonts
    \usepackage{mathpazo}
        
    % HEADER & FOOTER ________________________ %
    \usepackage{fancyhdr}
    \pagestyle{fancy}
    % header __________________
    \fancyhead{} % clear fields
%    \fancyhead[L]{TPE I}
%    \fancyhead[R]{2018.2}
    \setlength{\headheight}{16pt}    % tim's document says 46.445pt, but that's too much

    % \renewcommand{\headrule}{   % formating the header's line
    %    {\color{CPIorange}       % color
    %     \hrule                  % ?
    %     width\textwidth         % width
    %     height 2pt}  % thickness
    % }            

    % footer __________________
      \fancyfoot{} % clear fields
        \fancyfoot[L]{MPAM-ENAP}
        \fancyfoot[R]{\thepage}
        
        %\renewcommand{\footruleskip}{\baselineskip}
        \renewcommand{\footrule}{   % formating the footers's line
           {                        % color
            \hrule                  % ?
            width\textwidth         % width
            height 1pt}             % thickness
        }            
      \fancypagestyle{plain}{%
        %\renewcommand{\headrulewidth}{0pt}%
%        \fancyhead[L]{Técnicas de Pesquisa em Economia}
%        \fancyhead[R]{2018.2}
        
        \fancyfoot[L]{MPAM-ENAP}
        \fancyfoot[R]{\thepage}
      }        
        
    %__ APPENDIX _____________________________________________ %
    \usepackage[toc,page]{appendix}

%__ COMMANDS _________________________________________________________________________ %
    \newcommand{\mc}{\multicolumn}
    \newcommand{\lbar}{\underline}
    \newcommand{\ubar}{\overline}
    
    \DeclareMathOperator*{\argmax}{argmax}
    
    \newtheorem{theorem}{Theorem} %[section]
    \newtheorem{definition}{Definition}
    \newtheorem{remark}{Remark}
    \newtheorem{proposition}{Proposition}
    \newtheorem{condition}{Condition}
    \newtheorem{example}{Example}
    \newtheorem{lemma}{Lemma}
    \newtheorem{corollary}{Corollary}
    \newtheorem{assumption}{Assumption}
    \newtheorem{claim}{Claim}
    \newtheorem{fact}{Fact}
    
      \usepackage{listings}
  \usepackage{xcolor}
      \definecolor{BackgroundGray}{cmyk}{0,0,0,.1}
      \definecolor{CommentGreen}{cmyk}{.26,0,.50,.68}
      \definecolor{CPIgraydark}{RGB}{104,102,100}

  \lstset{ %
    backgroundcolor=\color{BackgroundGray}, % choose the background color; you must add \usepackage{color} or \usepackage{xcolor}
    basicstyle=\ttfamily\tiny,                 % the size of the fonts that are used for the code
    breakatwhitespace=false,              % sets if automatic breaks should only happen at whitespace
    breaklines=true,                      % sets automatic line breaking
    captionpos=b,                         % sets the caption-position to bottom
    commentstyle=\color{CommentGreen},    % comment style
    deletekeywords={...},                 % if you want to delete keywords from the given language
    escapechar={\@},                      % if you want to add LaTeX within your code
    extendedchars=true,                   % lets you use non-ASCII characters; for 8-bits encodings only, does not work with UTF-8
    frame=none,                           % adds a frame around the code
    keepspaces=true,                      % keeps spaces in text, useful for keeping indentation of code (possibly needs columns=flexible)
    keywordstyle=\color{blue},           % keyword style
    language=R,                           % the language of the code
    morekeywords={*,...},                 % if you want to add more keywords to the set
    morestring=[b]",
    morestring=[d]',
    numbers=none,                         % where to put the line-numbers; possible values are (none, left, right)
    numbersep=5pt,                        % how far the line-numbers are from the code
    numberstyle=\tiny\color{gray},        % the style that is used for the line-numbers
    rulecolor=\color{black},              % if not set, the frame-color may be changed on line-breaks within not-black text (e.g. comments (green here))
    showspaces=false,                     % show spaces everywhere adding particular underscores; it overrides 'showstringspaces'
    showstringspaces=false,               % underline spaces within strings only
    showtabs=false,                       % show tabs within strings adding particular underscores
    stepnumber=2,                         % the step between two line-numbers. If it's 1, each line will be numbered
    stringstyle=\color{red},              % string literal style
    tabsize=2,                            % sets default tabsize to 2 spaces
    title=\lstname,                       % show the filename of files included with \lstinputlisting; also try caption instead of title
    columns=flexible
  }
\lstset{literate=
  {á}{{\'a}}1 {é}{{\'e}}1 {í}{{\'i}}1 {ó}{{\'o}}1 {ú}{{\'u}}1
  {Á}{{\'A}}1 {É}{{\'E}}1 {Í}{{\'I}}1 {Ó}{{\'O}}1 {Ú}{{\'U}}1
  {à}{{\`a}}1 {è}{{\`e}}1 {ì}{{\`i}}1 {ò}{{\`o}}1 {ù}{{\`u}}1
  {À}{{\`A}}1 {È}{{\'E}}1 {Ì}{{\`I}}1 {Ò}{{\`O}}1 {Ù}{{\`U}}1
  {ä}{{\"a}}1 {ë}{{\"e}}1 {ï}{{\"i}}1 {ö}{{\"o}}1 {ü}{{\"u}}1
  {Ä}{{\"A}}1 {Ë}{{\"E}}1 {Ï}{{\"I}}1 {Ö}{{\"O}}1 {Ü}{{\"U}}1
  {â}{{\^a}}1 {ê}{{\^e}}1 {î}{{\^i}}1 {ô}{{\^o}}1 {û}{{\^u}}1
  {Â}{{\^A}}1 {Ê}{{\^E}}1 {Î}{{\^I}}1 {Ô}{{\^O}}1 {Û}{{\^U}}1
  {Ã}{{\~A}}1 {ã}{{\~a}}1
  {œ}{{\oe}}1 {Œ}{{\OE}}1 {æ}{{\ae}}1 {Æ}{{\AE}}1 {ß}{{\ss}}1
  {ç}{{\c c}}1 {Ç}{{\c C}}1 {ø}{{\o}}1 {å}{{\r a}}1 {Å}{{\r A}}1
  {€}{{\EUR}}1 {£}{{\pounds}}1
}  
  %\DefineShortVerb{\|}

\definecolor{darkblue}{rgb}{0.0, 0.0, 0.55}
\newcommand{\darkblue}[1]{{\color{darkblue} #1}}
